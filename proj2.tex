\documentclass[11pt, oneside]{article}   	% use "amsart" instead of "article" for AMSLaTeX format
\usepackage{geometry}                		% See geometry.pdf to learn the layout options. There are lots.
\geometry{letterpaper}                   		% ... or a4paper or a5paper or ...
%\geometry{landscape}                		% Activate for for rotated page geometry
%\usepackage[parfill]{parskip}    		% Activate to begin paragraphs with an empty line rather than an indent
\usepackage{graphicx}				% Use pdf, png, jpg, or eps with pdflatex; use eps in DVI mode
								% TeX will automatically convert eps --> pdf in pdflatex
\usepackage{amssymb}
\usepackage{amsmath}

\usepackage[numbered]{mcode}

\newcommand{\HRule}{\rule{\linewidth}{0.5mm}}

\title{A Birth-Death Process}
\setlength\parindent{0pt}

\begin{document}
\frenchspacing
\begin{titlepage}
		\begin{center}
			\includegraphics[scale=0.2]{logo}\\[1cm]

			\textsc{\LARGE MAT 485 Project 2}\\[2cm]
			\textsc{\Large Cal Poly Pomona}\\[1cm]


			\HRule \\[0.4cm]
			{\huge \bfseries A Birth-Death Process \\[0.4cm]}
			\HRule \\[2cm]

			\noindent
			\begin{minipage}{0.4\textwidth}
				\begin{flushleft}
					\large
					\emph{Authors:}\\
					Morgan Rupard \\ Aaron Gaut
				\end{flushleft}
			\end{minipage}
			\begin{minipage}{0.4\textwidth}
				\begin{flushright}
					\large
					\emph{Professor:}\\
					Dr. Jennifer Switkes
			\end{flushright}
			\end{minipage}

			\vfill

			{\large March $7^{\text{th}}$, 2016}
		\end{center}
	\end{titlepage}

\tableofcontents
\newpage

\section{Introduction}

In this project we will study two different birth-death models, a deterministic model:

$$\frac{dN}{dt} = (\lambda -\mu)N, \ N(0)=N_0, \ \lambda > 0, \ \mu > 0$$

and a stochastic model:

$$\frac{dP_n}{dt} = \mu (n+1)P_{n+1}+\lambda (n-1)P_{n-1}-(\lambda+\mu)nP_n, \ n>0$$

Here $\lambda$ is a birth rate, $\mu$ is a death rate, and $P_n$ is the probability that the population is size $n$ at time $t$.

\section{Deterministic Model}

In this section we will analyze the equation:

$$\frac{dN}{dt} = (\lambda -\mu)N, \ N(0)=N_0, \ \lambda > 0, \ \mu > 0$$

First if we solve this equation for the population size $N(t)$ we arrive at the following:

$$N(t) = N_0 e^{(\lambda-\mu)t}$$

We now will consider a case where $\lambda > \mu$, which should give us an exponential growth in our population because people are giving birth faster than people are dying.
We can confirm that via the plot below:

%\begin{center}
	%\includegraphics[]{} add later
%\end{center}

If we instead consider a case where $\mu > \lambda$ then we will see an exponential decay in our population because people are dying faster than people are giving birth.
Again we can confirm this with the following plot:

%\begin{center}
	%\includegraphics[]{} add later
%\end{center}

Now, if $\lambda = \mu$ then we will see that that the population will remain constant because people are being introduced into the population and being removed from the population at the same rate.
We see this in the plot below:

%\begin{center}
	%\includegraphics[]{} add later
%\end{center}

\section{Stochastic Model}

In this section we will explore the following equation:

$$\frac{dP_n}{dt} = \mu (n+1)P_{n+1}+\lambda (n-1)P_{n-1}-(\lambda+\mu)nP_n, \ n>0$$

We will analyze the expected value, $E[N(t)]$, and the variance, $V[N(t)]$, of this model in order to learn more about its behavior.
In general we can described the expected value as

$$E[N(t)] = \sum_{n=0}^\infty n P_n$$

If we go through some symbol pushing and algebra we can see that for this model the expected value is the following:

 $$E[N(t)] = N_0 e^{(\lambda-\mu)t}$$

 This is just our continuous model solution, which should not be too surprising.
 Next we need to find the variance for this model.
 We can do this by starting with the general equation of the variance.

 $$V[N(t)] = E[N(t)^2]-E[N(t)]^2$$

 By doing some more symbol pushing and algebra we can then see that the variance for this model is the following:

 $$V[N(t)] = N_0\left(e^{2(\lambda-\mu)t} - e^{(\lambda-\mu)t}\right)$$

 Below we will see graphically what these expressions tell us about the model.

 %\begin{center}
 	%\includegraphics[]{} add later
 %\end{center}

 \section{Simulations}
\end{document}
